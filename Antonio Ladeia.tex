\documentclass[12pt,a4paper]{article}
\usepackage[brazil]{babel}
\usepackage[utf8]{inputenc}
\usepackage{graphicx}
\usepackage{epsfig}
\oddsidemargin 0.46cm 
\textwidth 16.0cm 
\topmargin -0.77cm 
\textheight 24.7cm


\begin{document}

\begin{center} QUESTÕES \end{center}
\begin{enumerate}


\item Seja a matriz  
$\displaystyle 
B = \left[\begin{array}{cc}
cos(\theta)&-sen(\theta)\\
sen(\theta)&cos(\theta)
\end{array} \right]
$.

\begin{enumerate}

\item Mostre que  
$\displaystyle 
B^2 = \left[\begin{array}{cc}
cos(2 \theta) & -sen(2 \theta)\\
sen(2 \theta) & cos(2 \theta )
\end{array} \right]
$.  (Valor: 1,0). 

\item Determinar  
$\displaystyle 
B^n 
$.  (Valor: 1,5).
\end{enumerate}

\item Descrevendo as propriedades utilizadas, determinar:

\begin{enumerate}

\item a matriz \textbf{A}, sabendo que   
$\displaystyle 
\textbf{A}^{-1} = \left[\begin{array}{ccc}
1&0&\lambda\\
0&\alpha&0\\
\beta&0&\lambda
\end{array} \right]
$. (Valor: 1,5)

\item
$\displaystyle 
\lambda
$ de forma que 
$\displaystyle 
det(A - \lambda I) = -1
$. Sabendo-se que  
$\displaystyle 
A = \left[\begin{array}{ccc}1&0&2\\2&2&1\\1&2&-1\end{array}\right]
$ e 
$\displaystyle 
I 
$ a matriz identidade. (Valor: 1,5).
\end{enumerate}

\item Mostrar que a equação da circunferência que passa por três pontos
$\displaystyle P_1(x_1, y_1) $
, $\displaystyle P_2(x_2, y_2) $ 
e $\displaystyle P_3(x_3, y_3) $
no plano 
$\displaystyle xy $ é 
$\left|\begin{array}{cccc}
x^2+y^2&x&y&1\\
x_1^2+y_1^2&x_1&y_1&1\\
x_2^2+y_2^2&x_2&y_2&1\\
x_3^2+y_3^2&x_3&y_3&1
\end{array} \right|$
= 0. (Valor: 2,5).

\item Seja a relação linear
$\displaystyle \bf
\frac
{
	E(x_i-1) - 
    2E(x_i) + 
    E(x_i+1)
}
{
	\Delta x^2
}
+ 
\frac
{
    E(x_i)-E(x_i-1)
}
{
    \Delta x 
}
+  2E(x_i) = 3x_i^2
$.


Onde: $\displaystyle \bf \Delta x = x_i - x_{i-1}, E(x_0) = 0 $ e $\displaystyle \bf E(x_5) = 1 $.

Determinar os valores de $\displaystyle \bf E(x_i)$ para 
$\displaystyle \bf x = (1, 2, 3, 4)$. 
Apresentar os desenvolvimentos teórico e de cálculo. (Valor: 2,0).

\end{enumerate}
\end{document}
